\documentclass[11pt]{article}

\usepackage{hyperref}
\usepackage{biblatex}
\addbibresource{references.bib}

\begin{document}

\section{The Challenge of Negative Abatement Costs}

Surrey going net-zero (scopes 1 \& 2) by 2030\cite{oflynn2021}

A negative abatement cost refers to the cost of modifying a process which delivers a cost savings. As pointed out by \cite{elkins2011}, this represents a kind of market failure. Simple economic models assume that processes will automatically adjust to the most economically beneficial cost. 

For example, \cite{ukse-site} has a calculator which indicates a saving of 2.4m GBP over 16 years for a deal to buy 8GwH per year (roughly the usage of ARCHER -- needs confirmation)

Finding ways to reduce carbon emissions and save money at the same time is good for the budget, but poses a policy challenge: why have these measures not been implemented.

\section{Allocation versus Additionality}

The concept of allocation is used for calculating the division of teh footprint of a plant among many products.

Additionality, on the other hand, is used to assign an abatement amount for a given financial investment.

Additionality Definitions: 
\begin{itemize}
   \item \href{https://www.offsetguide.org/high-quality-offsets/additionality/}{Carbon Offset Guide}
   \item \href{https://ghginstitute.org/wp-content/uploads/2015/04/AdditionalityPaper_Part-1ver3FINAL.pdf}{Gillenwater 2012a}
   \item \href{https://www.researchgate.net/publication/257945160_What_Is_Additionality_Part_2_A_framework_for_a_more_precise_definition_and_standardized_approaches}{Gillenwater 2012b}
\end{itemize}

\printbibliography

\end{document}
